\documentclass{article}

% Language setting
% Replace `english' with e.g. `spanish' to change the document language
\usepackage[english]{babel}

% Set page size and margins
% Replace `letterpaper' with`a4paper' for UK/EU standard size
\usepackage[letterpaper,top=2cm,bottom=2cm,left=3cm,right=3cm,marginparwidth=1.75cm]{geometry}

% Useful packages
\usepackage{amsmath}
\usepackage{graphicx}
\usepackage[colorlinks=true, allcolors=blue]{hyperref}

\title{Visión y alcance }
\author{Judith Maldonado Garcia S19004911 }

\begin{document}
\maketitle

\begin{abstract}

501 ISW .
\end{abstract}

\section{Experiencia educativa : 
Pruebas de software. }

\section{Vision: }
La visión de este proyecto de software consiste en una formula matemática, en este caso la formula matemática utilizada sera para calcular el área de un cuadrado , mediante una pagina web. para poder facilitar este calculo , con tan solo poner un solo dato, se obtendrá el resultado.
\section{Alcance: }
Este proyecto de software, estará basado en  poder utilizar las tecnológicas y  como se menciono es una pagina web , se implementara el uso de html, hojas de estilos (css). y otras como lo es angular. y estará pasada por pruebas unitarias con el frame de jasmin. Ya que esto va ir acelerando los procesos de testeo.  Ya que esta materia se basa en cuantas pruebas se les puede hacer en un software ,para que este llegue a una funcionalidad libre de errores.  


\end{document}